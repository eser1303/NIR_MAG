\documentclass[12pt,a4paper]{article} %,fleqn
\usepackage{Diplo}
\usepackage[utf8]{inputenc} 
\usepackage[russian]{babel}

\usepackage[left=3cm,right=1.5cm,
top=2cm,bottom=2cm]{geometry}

\usepackage{comment}
\usepackage{hyperref}
\usepackage[square, comma, sort&compress, numbers]{natbib}
\newtheorem{definition}{Определение}
\newtheorem{theorem}{Теорема}
\newtheorem{utv}{Утверждение}

\graphicspath{{../pics/}}

\begin{document}
	
\vskip 3mm

\setcounter{page}{1}
%%%%%%%% Титульный лист %%%%%%%%%%%%%%%%%%%%%%%%%%%%%%%%%
\begin{center}
	\thispagestyle{empty}
	
	{ Министерство науки и высшего образования Российской Федерации\\}

	
	
	{ Московский Физико-технический институт \\}
	
	{ (Государственный Университет) \\}
	
	{ Физтех-школа прикладной математики и информатики  \\}
	
	{ Кафедра технологий цифровой трансформации\\[4cm]}
	
	
	{ \bf \Large Выпускная квалификационная работа\\}
	
	{ \bf \Large{"Развитие инструментов предиктивной аналитики в целях повышения эффективности мониторинга проектов в сфере жилищного строительства"\\[1cm]} }
	
	{\bf {Студента 2-го курса Ефремова Сергея Владимировича}\\[3cm]}
	
\end{center}

\begin{flushright}
	\bf{Научный руководитель}\\
	\bf{кандидат экономических наук, доцент Помулев А. А.}\\[4cm]
\end{flushright}


\begin{center}
	Москва, 2022
\end{center}
%%%%%%%%%%%%%%%%%%%%%%%%%%%%%%%%%%%%%%%%%%%%%%%%%%%%%%%%%%%%%%%%

\newpage
\begin{abstract}

Рассматривается задача улучшения инструментов предиктивной аналитики, использующихся при мониторинге проектов в сфере жилищного строительства. Исследованы предложенные ранее схемы решения этой проблемы, на основе изученных материалов разработан подход по улучшению оценки вероятности просрочки выплаты займа застройщиком на основании отчетности, публикуемой в открытом доступе и уровне зависимости от импортируемых комплектующих и материалов. Предложен, реализован и протестирован алгоритм, основанный на алгоритмах нейросетевого обучения с использованием чисел Шепли.	

\end{abstract}

\newpage
\tableofcontents
 


\newpage
\section{Введение}

На сегодняшний день можно с уверенностью говорить о том, что биометрические технологии идентификации вошли в нашу жизнь и используются во всех сферах человеческой деятельности, в том числе и в системах контроля. Существует целый ряд биометрических модальностей, позволяющих идентифицировать человека: рисунок папиллярных линий пальца, радужка глаза, походка, голос, почерк. Такое разнообразие методов объясняется тем, что при богатстве потенциальных сценариев использования биометрических модальностей, каждая из них может быть использована по-разному. Применение биометрических технологий уже активно осуществляется при проведении платежных операций в банковской сфере, особо нуждающейся в защите данных. Здесь они заменяют уже ставшие традиционными методы идентификации пользователей: ПИН-коды, пароли и т.д. Биометрические технологии имеют широкую перспективу использования, так как обладают целым рядом неоспоримых преимуществ, cреди которых можно выделить неотчуждаемость. Например, пароль может быть передан или украден третим лицом, и система все равно идентифицирует его как исходного пользователя, тогда как биометрическими модальностями труднее воспользоваться другому человеку или украсть без ведома обладателя. 

Использование биометрических систем настолько повсеместно, что обыватель даже не замечает, как часто и повсеместно ими пользуется. Так, для разблокировки мобильных устройств, применяется отпечаток пальца, снимок лица или радужки. Еще одним преимуществом биометрических систем считается удобство взаимодействия с ними пользователя. Их эффективность работы основывается на способности обработки поступающих данных с учетом изменчивости состояния и условий окружающей среды. То есть, даже в случае значительных искажений параметров входных данных, в связи с изменчивостью состояний (например, ухудшение условий освещенности, затемнение ресницами, отвод взгляда), системы биометрической идентификации, использующие изображения глаза, должны выдавать корректный результат и аутентифицировать пользователя адекватно. Также неоспоримо важным фактором для эффективности систем такого рода, является необходимость работы в условиях ограниченности не только вычислительных ресурсов, но и времени, требующегося на распознавание. 

Благодаря стремительному развитию технологий, с каждым днём все больше расширяется сфера практического применения технических средств цифровой обработки изображений. Одним из направлений стало распознавание человека на основе изображения глаза~(рис.~\ref{fig:glaz}). Эта идея не нова, ещё в 1936 году офтальмологи \cite{Medic} предложили использовать радужную оболочку глаза для распознавания. Далее исследователями всего мира разрабатывались различные методы распознавания человека по его радужной оболочке, но по-настоящему сильный рост интереса произошел только в начале XXI века. Это объясняется качественными изменения условий жизни общества: эволюция средств регистрирования и обработки изображений, необходимость распознавания личности в связи с террористической угрозой~\cite{Conf3, Terr, Tech}. Наряду с идентификацией обработка изображения радужки применяется в медицинских целях, где также важно выделение области интереса. 

\begin{figure}[h]
	
	\centering
	
	\includegraphics[width=0.6\linewidth]{glaz.jpg}
	
	\caption{Внешнее строение глаза.}
	
	\label{fig:glaz}
	
\end{figure}

Однако получить качественное и обособленное изображение радужки на практике не так просто, и важной задачей оказывается необходимость корректно определять границы глаза, задаваемые веками. Исследования в этом направлении породили различные модели, которые также используют в биометрических системах распознавания, поскольку уточнение границ и исключение нерелевантных областей оказывает значительное влияние на точность результата.

В ходе данного исследования разработан метод выделения века как параметрической кривой на изображении глаза. Он состоит в определении параметров кривой максимизацией суммы вертикального градиента яркости, рассматриваемого в окрестности века.
 

\newpage
\subsection{Цели и задачи работы}

\textbf{Цель и задачи исследования.} Целью исследования является построение модели предиктивной аналитики, которая позволит повысить эффективность процесса мониторинга проектов коммерческим банком в сфере жилищного строительства и улучшить качество прогнозирования вероятности просрочки платежа по сравнению с существующими моделями. Для реализации этой цели были поставлены следующие задачи:
\begin{itemize}
	\item изучить определение понятий: «мониторинг», «эффективность мониторинга», «предиктивная аналитика» для использования в настоящем исследовании;
	\item провести анализ существующих проектов и динамики их развития в сфере жилищного строительства;
	\item ознакомиться с текущим состоянием финансирования проектов в сфере жилищного строительства и нормативно-правовой базой;
	\item изучить процесс мониторинга проектов и методы их оценки коммерческим банком;
	\item исследовать типы моделей предиктивной аналитики и их применение в кредитном процессе;
	\item выделить основные проблемы процесса мониторинга проектов и определить возможности их решения с использованием инструментария предиктивной модели 
	\item разработать алгоритм внедрения разработанного инструментария в бизнес-процесс мониторинга
	\item рассчитать экономический эффект от внедрения модели 
\end{itemize}

\bigskip

\textbf{Научная новизна.} Используется нейросетевой подход к определению вероятности банкротства заемщика с выделением признаков, вносящих максимальный вклад с помощью, чисел Шепли. В работе предлагается коэффициент, позволяющий оценить зависимость застройщика от импортных комплектующих и материалов, а также уровень потенциального риска, обусловленного политическими ограничениями.

\bigskip

\textbf{Методы исследования.} Алгоритмы реализованы на языке программирования Python с использованием библиотек |||.

\bigskip

\textbf{Практическая ценность.} Полученная модель может быть использована в качестве встраиваемого модуля. Например, с её помощью можно:
\begin{itemize}
	\item корректировать оценку вероятности просрочки платежа застройщиком, учитывая его зависимость от импортируемых компонентов;
	\item дополнять существующие системы мониторинга объектов строительства показателем уровня зависимости от импортных компонентов и моделью оценки наиболее важных показателей, влияющих на просрочку.
\end{itemize}
%%%%%%%%%%%%%%%%%%%%%%%%%%%%%%%%%%%%%%%%%%%%%%%%%%%%%


\newpage
\section{Постановка задачи}


\subsection{Мониторинг проектов}\label{task}

изучить определение понятий: «мониторинг», «эффективность мониторинга», «предиктивная аналитика» для использования в настоящем исследовании

Мониторинг проекта - процесс измерения показателей выполнения проекта, сбора данных об исполнении проекта, информационного обслуживания управления проектом с целью выявления его соответствия желаемому результату и плану, с последующим представлением и распространением полученных данных.

Под контролем проекта понимается процесс сравнения фактических значений контрольных показателей с запланированными, последующего анализа отклонений, оценки тенденций и прогнозирования возможных альтернатив, разработки корректировок хода реализации проекта для улучшения прогноза.

Основными целями контроля и мониторинга инвестиционных проектов можно считать обеспечение:
\begin{itemize}
	\item своевременного достижения целей проекта с учетом согласованной стоимости;
	\item срочности, возвратности, платности и целевого использования предоставляемых банком кредитных ресурсов для финансирования проекта;
	\item своевременного информирования руководства банка о выявленных проблемах, прогнозирования рисков реализации проекта и разработка мер по их снижению;
	\item достижения заложенных в проекте показателей социально-экономической эффективности.  
\end{itemize}

Чаще всего при реализации проектов в сфере жилищного строительства выделяют следующие виды мониторинга:
\begin{itemize}
	\item мониторинг хода реализации инвестиционного проекта (сроков выполнения работ, бюджета проекта, расчетного времени окончания работ и расчетной стоимости проекта, организация технадзора и контрольных проверок);
	\item финансовый мониторинг (финансово-экономического состояния заемщика, исполнителя проекта, поручителей, гарантов, обеспечения по кредиту/кредитной линии; денежного потока, коэффициентов покрытия, целевого использования средств, исполнения заемщиком обязательств перед банком);
	\item мониторинг эффективности инвестиционного проекта (показателей, которые предусмотрены положением об экспертизе проектов банка).
\end{itemize}

Такое разделение обуславливается необходимостью не только контролировать текущую операционную деятельность, ведущуюся по проекту, исполнение финансовых обязательств участниками проекта и целевое использование средств, но и конечные результаты этой деятельности, которые выражаются в достижении целей проекта и достигнутой социально-экономической значимости.

Основными элементами систем мониторинга инвестиционных проектов являются:
\begin{itemize}
	\item финансовая, техническая и иная отчетность заемщика;
	\item экспертные оценки банковских специалистов по направлениям реализации проекта, независимые эксперты (технический надзор, финансовый аудит);
	\item календарно-сетевые графики работ, расчеты сроков ввода объекта в эксплуатацию и суммарной стоимости работ;
	\item данные автоматизированных информационных систем мониторинга инвестиционных проектов.
\end{itemize}

Последние и будут рассмотрены в первую очередь в данной работе.

Ключевые этапы мониторинга проектов:
\begin{itemize}
	\item этап подготовки проекта (начинается с момента одобрения займа/кредитной линии и заканчивается выделением финансовых средств);
	\item инвестиционная стадия проекта (непосредственное финансирование проекта);
	\item этап эксплуатации (следует до полного исполнения заемщиком платежных обязательств перед банком).
\end{itemize}

\subsection{Эффективность мониторинга}

Построением эффективных систем мониторинга занимались многие исследователи Д. Боуэр, Дж.Филлипс, Р.Фартел, Х.Керцнер[здесь будут ссылки на литературу]. Мониторинг в современных реалиях представляет из себя комплексную функцию проектного управления, в которую входит процедуры сбора, анализа и передачи информации о ходе реализации проекта, которая позволяет решить проблему своевременного принятия решений по проекту.

Основные задачи, которые решают системы мониторинга:
\begin{itemize}
	\item определение совокупности отслеживаемых индикаторов;
	\item организация обработки и агрегирования полученной информации;
	\item генерация текущей отчетности по проекту;
	\item интеграция функции мониторинга в информационную архитектуру предприятия, реализующего проект.
\end{itemize}

Принятие управленческих решений о формировании и развитии системы мониторинга проектов,о требуемом кадровом, техническом и финансовом обеспечении неизбежно связано с дополнительными затратами. Однако, потенциальные угрозы от финансирования убыточных или высокорисковых проектов также способны привести к значительным издержкам. Все это остро ставит вопрос о необходимости эффективного мониторинга проектов.

Основными подходами к изучению эффективности проектов являются:
\begin{itemize}
	\item целевой (предполагает анализ степени достижения целевых значений показателей);
	\item динамический (учитывает скорость изменения исследуемых показателей во времени и относительно друг друга);
	\item затратный (основан на сопоставлении затрат и результатов);
	\item ресурсный (исследует степень рациональности расходования ресурсов).
\end{itemize}

\subsection{Предиктивная аналитика}

Предиктивной аналитикой или продвинутой аналитикой называют ряд аналитических и статистических методов прогнозирования действий и поведения в будущем. В основе лежат статистические модели, позволяющие находить закономерности в исторических и транзакционных данных, что позволяет выделять потенциальные риски и возможности. Ключевые этапы составляющие процесс предиктивного анализа: подключение к данным, анализ и визуализация результатов исследований, развитие предложений и моделей данных, применение предиктивных моделей, оценка и прогнозирование будущих результатов.

В основе предиктивной аналитики лежит выявление связей между данными историческими и прогнозными результатами на их основе. Верхнеуровнево алгоритмы предиктивного анализа можно разделить на контролируемое и неконтролируемое обучение.

Контролируемое обучение принято разделять на две ключевые категории: регрессию для количественных ответов и классификацию для определения фактической принадлежности ответа к той или иной группе. 
 
Неконтролируемое обучение применяется для получения выводов из входных данных без разметки. Наиболее распространенный вид такого анализа - кластеризация, которую используют для поиска скрытых закономерностей в данных.

%%%%%%%%%%%%%%%%%%%%%%%%%%%%%%%%%%%%%%%%%%%%%%%%%%%%%

\newpage
\section{Обзор действующей практики}
\subsection{Анализ существующих проектов}

\subsubsection{Проект 1}


\subsubsection{Проект 2}

\subsection{Текущее состояние финансирования в сфере жилищного строительства}

\subsubsection{Ключевой топик 1}


\subsubsection{Ключевой топик 2}

\subsection{Процесс мониторинга проектов и методы оценки коммерческим банком}

\subsection{Типы моделей предиктивной аналитики и их применение в кредитном процессе}

%%%%%%%%%%%%%%%%%%%%%%%%%%%%%%%%%%%%%%%%%%%%%%%%%%%%%%

\newpage
\section{Формальная постановка задачи}

\subsection{Ключевые проблемы процесса мониторинга проектов}

\subsection{Возможности внедрения с учетом консервативности и систем безопасности банка}


%%%%%%%%%%%%%%%%%%%%%%%%%%%%%%%%%%%%%%%%%%%%%%%%%%%%%%

\newpage
\section{Описание модели}
 
\subsection{Этап предобработки данных}
 
\subsection{Ядро модели}

%%%%%%%%%%%%%%%%%%%%%%%%%%%%%%%%%%%%%%%%%%%%%%%%%%%%%

\newpage
\section{Результаты работы алгоритма}
\subsection{Пример полученных результатов - ключевые атрибуты}

\subsection{Сравнение результатов с другими методами}

\subsection{Сравнение результатов с оценкой предложенной метрики качества}


%%%%%%%%%%%%%%%%%%%%%%%%%%%%%%%%%%%%%%%%%%%%%%%%%%%%%

\newpage
\section{Экономический эффект от внедрения модели}

%%%%%%%%%%%%%%%%%%%%%%%%%%%%%%%%%%%%%%%%%%%%%%%%%%%%%


\newpage
\section{Заключение}



%%%%%%%%%%%%%%%%%%%%%%%%%%%%%%%%%%%%%%%%%%%%%%%%%%%%%


\newpage

\bibliographystyle{gost71s}
\bibliography{mylib}
\nocite{GV}

\end{document}